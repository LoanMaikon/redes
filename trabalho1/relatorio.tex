\documentclass[a4paper, 12pt]{article}
\usepackage[margin=2cm]{geometry}
\usepackage{indentfirst}
\usepackage[utf8]{inputenc}
\usepackage{hyperref}

\title{\textbf{Relatório T1 Redes}}
\author{Bruno Aquiles de Lima \\ Luan Marko Kujavski}
\date{}

\begin{document}
\maketitle

\section{Introdução}
O trabalho consiste em implementar uma plataforma de ``streaming" de vídeo, onde o 
cliente pode fazer requisições para listar os vídeos disponíveis e para assisti-los.
O link para o repositório do projeto é \url{https://github.com/LoanMaikon/redes/tree/main/trabalho1}

\section{Funcionamento dos programas}
Após rodar \texttt{make}, são gerados os executáveis \texttt{server} e \texttt{client}. 
O uso dos programas é da seguinte forma: \texttt{sudo ./server <interface>}. Para o
\texttt{client}, o uso é exatamente o mesmo.

\subsection{Funcionamento do Client}
Ao rodar o \texttt{client}, são apresentadas as seguintes opções:
\begin{enumerate}
    \setlength\itemsep{-0.5em}
    \item[1-]\texttt{Mostrar vídeos} \textit{\small{- Lista os vídeos disponíveis no servidor e seus respectivos IDs}}
    \item[2-]\texttt{Baixar arquivo} \textit{\small{- Baixa um arquivo do servidor com base no ID do vídeo}}
    \item[3-]\texttt{Sair} \textit{\small{- Encerra a execução do programa}}
\end{enumerate}

\subsection{Funcionamento do Server}
\begin{itemize}
    \item\textbf{Inicialização:} O servidor lista os vídeos no diretório \texttt{movies/}
        e armazena os nomes em uma estrutura \texttt{movies\_t}, que contém um vetor de strings
        e um \texttt{unsigned long int} para o tamanho do vetor.

    \item\textbf{Loop Principal:} O servidor entra em um loop aguardando requisições
        de listagem e download. Requisições de outros tipos são ignoradas.

    \item\textbf{Requisição de Listagem:} Quando uma requisição de listagem é recebida,
        os nomes dos vídeos são enviados ao cliente na ordem definida pela
        estrutura \texttt{movies\_t}.
\end{itemize}

\section{Decisões de implementação}

\subsection{Timeout}
No socket, foi definido o tempo de espera
da função \texttt{recv} para 300 ms, assim o processo não fica travado esperando um pacote.
Assim, foi possível a construção do timeout usando a função \texttt{time()}.

\subsection{Implementação janela deslizante}

\subsubsection*{Cliente:}
\begin{itemize}
    \item Mantém um vetor server\_packets de 5 pacotes (buffers de 67 bytes).
Recebe até 5 pacotes, valida a sequência e corrige pacotes fora de ordem, mantendo
uma variável para controlar a sequência correta dos pacotes.
\end{itemize}

\subsubsection*{Servidor:}
\begin{itemize}
    \item Utiliza uma lista ligada window\_packet\_t para armazenar pacotes. Cada nodo contém um
    pacote e um ponteiro para o próximo nodo, e o cabeçalho window\_packet\_head\_t contém ponteiros
    para o primeiro e o último nodo e um unsigned long int para o tamanho da lista.

    \item O servidor lê arquivos em blocos de 63 KB, quebra-os em pacotes, e os organiza na lista 
    ligada. Se a quantidade de pacotes restantes for menor que o tamanho da janela, os pacotes antigos
    são desalocados, e novos pacotes são lidos e concatenados no final da lista. 
    O primeiro nodo da nova leitura tem uma sequência incrementada em relação ao último pacote 
    do bloco anterior.

    \item A lista ligada facilita a remoção de pacotes antigos e a adição de novos pacotes,
        mantendo a janela de pacotes sempre preenchida.
\end{itemize}

\subsection{Tamanho dos pacotes}
O tamanho de todos os pacotes ficou fixo em 67 bytes. Incialmente, foi implementado
com tamanho dinâmico, porém, tornou o código mais complexo e não compensava a economia,
pois apenas o último pacote de cada bloco de 63 KB teria tamanho diferente.

\subsection{Escape para bytes problemáticos}
Durante a construção dos pacotes de dados, para todo byte 0x81 e 0x88, é adicionado em seguida
um byte 0xFF para evitar problemas de interpretação. Não é feito esse tratamento no momento
do envio dos pacotes com os nomes dos vídeos, pois o intervalo dos bytes dos caracteres é 
0x20 até 0x7E.

\end{document}
