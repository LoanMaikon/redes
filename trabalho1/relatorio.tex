\documentclass[a4paper, 12pt]{article}
\usepackage[margin=2cm]{geometry}
\usepackage{indentfirst}
\usepackage[utf8]{inputenc}
\usepackage{hyperref}

\title{\textbf{Relatório T1 Redes}}
\author{Bruno Aquiles de Lima \\ Luan Marko Kujavski}
\date{}

\begin{document}
\maketitle

\section{Introdução}
O trabalho consiste em implementar uma plataforma de ``streaming" de vídeo, onde o 
cliente pode fazer requisições para listar os vídeos disponíveis e para assistir a
um vídeo. O link para o repositório do projeto é \url{https://github.com/LoanMaikon/redes/tree/main/trabalho1}

\section{Decisões de implementação}
Após rodar \texttt{make}, são gerados os executáveis \texttt{server} e \texttt{client}. 
O uso dos programas é da seguinte forma: \texttt{sudo ./server <interface>}. Para o
\texttt{client}, o uso é exatamente o mesmo.

\subsection{Client}
Ao rodar o \texttt{client}, são apresentadas as seguintes opções:
\begin{enumerate}
    \setlength\itemsep{-0.5em}
    \item[1-]\texttt{Mostrar vídeos} \textit{\small{- Lista os vídeos disponíveis no servidor e seus respectivos IDs}}
    \item[2-]\texttt{Baixar arquivo} \textit{\small{- Baixa um arquivo do servidor com base no ID do vídeo}}
    \item[3-]\texttt{Sair} \textit{\small{- Encerra a execução do programa}}
\end{enumerate}

\subsection{Server}
O primeiro passo do \texttt{server} é listar os vídeos disponíveis no diretório \texttt{movies/} e
armazenar seus nomes em uma estrutura chamada \texttt{movies\_t}, a qual é composta de um vetor de
strings e um \texttt{unsigned long int} para o tamanho do vetor. Em seguida, o servidor entra em um
loop principal onde aguarda requisições do cliente.

No loop principal, são esperadas requisições de listagem e download, outros tipos de pacotes são ignorados.

Quando recebe um pedido de listagem, os nomes dos vídeos são enviados usando ``para-espera".
A ordem de envio é com base nas informações contidas na estrutura \texttt{movies\_t}.

\end{document}

